\documentclass{article}
\PassOptionsToPackage{numbers,compress}{natbib}
\usepackage[final]{template22}
% Common packages
\usepackage[utf8]{inputenc} % allow utf-8 input
\usepackage[T1]{fontenc}    % use 8-bit T1 fonts
\usepackage{microtype}
\usepackage{times}
\usepackage{graphicx}
\usepackage{amsmath,amssymb,mathbbol}
% \usepackage{algorithmic}
% \usepackage[linesnumbered,ruled,vlined]{algorithm2e}
\usepackage{acronym}
\usepackage{enumitem}
\usepackage[pagebackref=true,breaklinks=true,colorlinks]{hyperref}
\usepackage{balance}
\usepackage{xspace}
\usepackage{setspace}
\usepackage[skip=3pt,font=small]{subcaption}
\usepackage[skip=3pt,font=small]{caption}
\usepackage[dvipsnames]{xcolor}
\usepackage[capitalise]{cleveref}
\usepackage{booktabs,tabularx,colortbl,multirow,array,makecell}
% \usepackage{overpic,wrapfig}

\usepackage{fancyhdr}
\hypersetup{pdfencoding=auto,colorlinks=true,allcolors=black}
\renewcommand{\headrulewidth}{0.5pt}
\renewcommand{\footrulewidth}{0pt}
\fancyhf{}
\pagestyle{fancy}
\fancyhead[L]{\runningauthor}
\fancyhead[R]{Tech Report PKU-IAI-\runningid}
\fancyfoot[C]{\thepage}

% Handy shorthand
\makeatletter
\DeclareRobustCommand\onedot{\futurelet\@let@token\@onedot}
\def\@onedot{\ifx\@let@token.\else.\null\fi\xspace}
\def\eg{\emph{e.g}\onedot} 
\def\Eg{\emph{E.g}\onedot}
\def\ie{\emph{i.e}\onedot} 
\def\Ie{\emph{I.e}\onedot}
\def\cf{\emph{c.f}\onedot} 
\def\Cf{\emph{C.f}\onedot}
\def\etc{\emph{etc}\onedot} 
\def\vs{\emph{vs}\onedot}
\def\wrt{w.r.t\onedot} 
\def\dof{d.o.f\onedot}
\def\etal{\emph{et al}\onedot}
\makeatother

\definecolor{gray}{gray}{0.9}

% Handy math ops
\DeclareMathOperator*{\argmax}{arg\,max}
\DeclareMathOperator*{\argmin}{arg\,min}
\newcommand{\norm}[1]{\left\Vert #1 \right\Vert}

% % Spacing
\frenchspacing
% \medmuskip=2mu   % reduce spacing around binary operators
% \thickmuskip=3mu % reduce spacing around relational operators
% \setlength{\abovedisplayskip}{3pt}
% \setlength{\belowdisplayskip}{3pt}
% \setlength{\abovecaptionskip}{3pt}
% \setlength{\belowcaptionskip}{3pt}
\setlength\floatsep{0.5\baselineskip plus 3pt minus 2pt}
\setlength\textfloatsep{0.5\baselineskip plus 3pt minus 2pt}
\setlength\dbltextfloatsep{0.5\baselineskip plus 3pt minus 2pt}
\setlength\intextsep{0.5\baselineskip plus 3pt minus 2pt}

\makeatletter
\renewcommand{\paragraph}{%
  \@startsection{paragraph}{4}%
  {\z@}{0ex \@plus 0ex \@minus 0ex}{-1em}%
  {\hskip\parindent\normalfont\normalsize\bfseries}%
}
\makeatother

% Graphics path
\graphicspath{{figures/}}

% Clever references
\crefname{algorithm}{Alg.}{Algs.}
\Crefname{algorithm}{Algorithm}{Algorithms}
\crefname{section}{Sec.}{Secs.}
\Crefname{section}{Section}{Sections}
\crefname{table}{Tab.}{Tabs.}
\Crefname{table}{Table}{Tables}
\crefname{figure}{Fig.}{Fig.}
\Crefname{figure}{Figure}{Figure}

% Acronym
\acrodef{pku}[PKU]{Peking University}

\title{Beyond Recursion: Building a Non-Recursive Model for Bayesian Social Cognition}
\newcommand{\runningauthor}{Yuchen-Wang}
\newcommand{\runningid}{2100013153}

\author{%
  Yuchen Wang\thanks{https://wangyuchen333.github.io/} \\
  Department of EECS\\
  Peking University\\
  \texttt{wangyuchen333@stu.pku.edu.cn} \\
}

\begin{document}
\maketitle

\begin{abstract}
The Theory of Mind (ToM) has traditionally been explored through recursive modeling, which involves attributing mental states to others based on their beliefs, desires, and intentions. However, this approach faces limitations due to its computational intensity and cognitive demand. This paper proposes a non-recursive model that incorporates social cognition and Bayesian inference, aligning with the observed efficiency of human social cognition. We consider the role of shared experiences and joint attention in building a common mind during social interactions. The paper concludes that a more feasible model for human social cognition may involve simpler heuristic strategies, probabilistic reasoning, and the establishment of common ground, avoiding the complexities and potential errors of recursive modeling.
\end{abstract}


\section{Introduction}
The traditional recursive model of Theory of Mind (ToM) posits that individuals attribute mental states to others by assuming beliefs, desires, and intentions that explain observed actions \citep{premack1978does}. This model has been instrumental in understanding social cognition; however, it is not without limitations. Humans do not engage in infinite recursive mind inference in practice, suggesting that the model may be overly complex and cognitively demanding \citep{doshi2020recursively}. The primary issue with recursive modeling is its computational intensity and the cognitive load it imposes. It assumes that individuals continuously update their beliefs about others' beliefs, which can lead to exponential growth in complexity as each level of recursion is added \citep{doshi2020recursively}. This process can become unwieldy and may not align with the observed efficiency of human social cognition \citep{royzman2003know}.

\section{The Limitations of Recursive Modeling}
Recursive modeling's assumption that individuals continuously update their beliefs about others' beliefs leads to a significant cognitive burden. As each level of recursion is added, the complexity grows exponentially, making the process impractical for everyday social interactions \citep{doshi2020recursively}. This complexity suggests that recursive modeling may not accurately represent the cognitive processes underlying human social cognition, which is typically more efficient and less resource-intensive \citep{royzman2003know}.

\section{Building a Common Mind Through Shared Experiences}
Regarding the construction of a common mind during interaction, humans may rely on shared experiences, mutual understanding, and the development of joint attention \citep{Tomasello2018}. This process involves the coordination of attention and the establishment of common ground, which allows individuals to align their mental states and build a shared understanding without the need for deep recursive inferences \citep{Sebanz2003}.

\subsection{Social Cognition}

Over the arc of evolution, humans have honed a sophisticated array of social cognitive skills that are essential for participating in intricate cultural practices and for skillfully maneuvering through the subtleties of social interactions. These specialized capabilities endow us with the ability not only to comprehend but also to foresee the actions of others, facilitating smooth communication and collaboration within our social milieu. As a core component of the Theory of Mind (ToM) framework, social cognition highlights the imperative to develop a ToM that is rooted in social cognition, one that can effectively curtail the recursive process \citep{herrmann2007humans}. Social cognition empowers agents to anticipate and apply common sense, thereby halting recursion, and similarly, enables agents to recognize when a conclusion is based on a false belief, effectively terminating it, akin to the cognitive development observed in infants \citep{houde2022cambridge}.

A more viable model for the mental mechanisms at play incorporates an approach that utilizes simpler, heuristic-based strategies for mind-reading \citep{murphy2009beyond}. This model resonates more profoundly with the observed efficiency of human social cognition and the bounded rationality that is a hallmark of decision-making processes \citep{Simon1997}. By adopting such an approach, we can more accurately capture the practicalities of how humans traverse the social landscape, avoiding the entanglement of infinite recursive thought loops.

\subsection{Bayesian Inference}
Evolutionary theory not only elucidates the origins of human cognitive abilities such as language, metacognition, and spatial reasoning but also provides a mathematical framework for charting the evolution of thoughts, perspectives, and memories within the individual mind. Bayesian inference serves as a powerful tool for agents to predict each other's Theory of Mind (ToM), thereby facilitating consensus more readily \citep{suchow2017evolution}.

Incorporating Bayesian inference into the model of ToM offers an alternative approach that permits the updating of beliefs regarding others' mental states based on observed behaviors and contextual information \citep{Khalvati2019}. This method steers away from recursive processes and instead concentrates on the probabilistic estimation of others' mental states, which may more accurately mirror the cognitive processes that humans employ in social interactions \citep{rabinowitz2018machine}. By adopting this probabilistic stance, we can better understand how humans navigate the complexities of social cognition without being bogged down by infinite recursive loops of inference. This shift towards a more dynamic and probabilistic model of social cognition aligns with the observed efficiency and bounded rationality in human decision-making processes, enhancing our understanding of how consensus is formed and maintained in social groups.




\section{Conclusion}
In conclusion, a more feasible model for human social cognition may involve a combination of simpler heuristic strategies and probabilistic reasoning, alongside the establishment of shared experiences and common ground during interaction. This non-recursive approach, which includes elements of social cognition and Bayesian inference, is more aligned with the observed efficiency of human social cognition and the bounded rationality observed in decision-making processes. By relying on shared experiences, mutual understanding, and the development of joint attention, humans can align their mental states and build a shared understanding without the need for deep recursive inferences, thus avoiding the computational complexity and potential errors associated with recursive modeling.



\bibliographystyle{plainnat}
\bibliography{reference}


\end{document}